\documentclass[tikz,border=6mm]{standalone}
\usepackage{fontspec}
\usepackage{xeCJK}
\setmainfont{DejaVu Sans}
\setCJKmainfont{WenQuanYi Zen Hei}
\usepackage{tikz}
\usetikzlibrary{calc,arrows.meta,positioning}

\begin{document}
\begin{tikzpicture}[x=1cm,y=1cm]
  \def\W{30}
  \def\H{20}

  % 背景:金属哑光 + 生物荧光点
  \path[rounded corners=18pt, top color=black!85!gray!40, bottom color=black!80!blue!40]
    (0,0) rectangle (\W,\H);
  \foreach \i in {0,...,90} {
    \fill[cyan!50!lime!70, opacity=0.08, rounded corners=2pt]
      ({rnd*30}, {rnd*20}) circle ({0.04+0.08*rnd});
  }
  \path[fill=white, fill opacity=0.03, rounded corners=16pt, draw=white!10]
    (0.7,0.7) rectangle (\W-0.7,\H-0.7);

  % 斜向体积光条
  \path[fill=cyan!40, opacity=0.09, rotate=12] (-2,8) rectangle (20,10);
  \path[fill=magenta!30, opacity=0.07, rotate=-10] (10,2) rectangle (34,4);

  % Isometric 线框块
  \foreach \k in {0,...,4} {
    \pgfmathsetmacro{\x}{18+1.8*\k}
    \pgfmathsetmacro{\y}{4+0.9*\k}
    \path[draw=white!35, fill=white, fill opacity=0.04, line width=0.4pt]
      (\x,\y) -- ++(1.6,0.8) -- ++(0,1.4) -- ++(-1.6,-0.8) -- cycle;
    \path[draw=white!35, fill=white, fill opacity=0.05, line width=0.4pt]
      (\x,\y) -- ++(1.6,0.8) -- ++(1.0,-0.6) -- ++(-1.6,-0.8) -- cycle;
  }

  % 标题区
  \node[anchor=west, text=white, font=\bfseries\fontsize{22pt}{24pt}\selectfont] at (1.2,18.4)
    {文心墨韵|VR/AR × AI艺术评测海报·异形视角};
  \node[anchor=west, text=white!85, font=\fontsize{11pt}{13pt}\selectfont] at (1.25,17.5)
    {Isometric view, Metallic + Matte finish, Bioluminescence, Soft studio lighting};
  \node[anchor=east, text=white!80, font=\fontsize{9pt}{11pt}\selectfont] at (\W-1.2,18.4)
    {3:2|30\,cm×20\,cm|XeLaTeX + TikZ};

  % 左列:平台概览
  \path[rounded corners=10pt, fill=white, fill opacity=0.07, draw=white!30]
    (1.2,5.2) rectangle (9.5,16.6);
  \node[anchor=north west, text=white, font=\bfseries\fontsize{14pt}{16pt}\selectfont] at (1.6,16.4)
    {平台总览};
  \node[anchor=north west, text=white!90, font=\fontsize{10pt}{13pt}\selectfont] at (1.6,15.8) {\begin{minipage}{7.6cm}
    • 42 模型/15 机构,28 真实模型上线;评分、雷达与对战可视化。\\[2pt]
    • 前端 React 19 + Tailwind(iOS + Glassmorphism 可选),文本高亮/切块。\\[2pt]
    • 后端 FastAPI + SQLAlchemy,统一模型接口,WebSocket 对战。\\[2pt]
    • 部署:GCP Cloud Run / Cloud SQL / Storage,CI/CD 15 分钟。\\[2pt]
    • 页面:排行榜 / 模型详情 / 对战 / 评测任务 / 登录;52 组件,64 E2E。
  \end{minipage}};

  % 左列:VR 案例
  \path[rounded corners=8pt, fill=white, fill opacity=0.08, draw=white!25]
    (1.2,2.2) rectangle (9.5,4.8);
  \node[anchor=west, text=cyan!20, font=\bfseries\fontsize{10pt}{12pt}\selectfont] at (1.5,4.5)
    {VR/AR 案例|Isometric 手触线框};
  \node[anchor=west, text=white!88, font=\fontsize{9pt}{12pt}\selectfont] at (1.5,3.9)
    {提示词:Wireframe mesh + Metallic/Matte,Holography,Rule of thirds};
  \node[anchor=west, text=white!70, font=\fontsize{8pt}{10pt}\selectfont] at (1.5,3.1)
    {可加:Isometric view, Knolling, Soft studio lighting, Bioluminescence};

  % 中列:架构流
  \path[rounded corners=10pt, fill=white, fill opacity=0.07, draw=white!30]
    (10.2,5.2) rectangle (19.6,16.6);
  \node[anchor=north west, text=white, font=\bfseries\fontsize{14pt}{16pt}\selectfont] at (10.6,16.4)
    {端到端架构};
  \node[rounded corners=6pt, fill=cyan!30!white, draw=white!70, text=black, font=\bfseries] (user) at (13,14.8) {用户 / 游客};
  \node[rounded corners=6pt, fill=white, fill opacity=0.9, text=black, font=\bfseries] (front) at (13,12.8)
    {前端 React19 + Tailwind};
  \node[rounded corners=6pt, fill=white, fill opacity=0.86, text=black, font=\bfseries] (backend) at (13,10.8)
    {FastAPI 服务层};
  \node[rounded corners=6pt, fill=white, fill opacity=0.86, text=black, font=\bfseries] (umi) at (13,8.8)
    {统一模型接口 + 评测引擎};
  \node[rounded corners=6pt, fill=white, fill opacity=0.86, text=black, font=\bfseries] (providers) at (13,6.8)
    {8 家模型 Provider};
  \draw[-{Latex[length=4mm]}, thick, white!75] (user) -- (front);
  \draw[-{Latex[length=4mm]}, thick, white!75] (front) -- node[right, text=white!75, font=\fontsize{8pt}{10pt}\selectfont] {JWT/游客双认证} (backend);
  \draw[-{Latex[length=4mm]}, thick, white!75] (backend) -- node[right, text=white!75, font=\fontsize{8pt}{10pt}\selectfont] {任务队列 / WebSocket} (umi);
  \draw[-{Latex[length=4mm]}, thick, white!75] (umi) -- node[right, text=white!75, font=\fontsize{8pt}{10pt}\selectfont] {OpenAI / Anthropic / DeepSeek / Qwen ...} (providers);

  \node[rounded corners=6pt, fill=white, fill opacity=0.06, draw=white!25, text=white, font=\fontsize{9pt}{12pt}\selectfont, align=left] at (16.8,9.8) {多阶段评测\\• 分析提示词\\• 生成/润色\\• 质量评估\\• 结果可视化};
  \node[rounded corners=6pt, fill=white, fill opacity=0.06, draw=white!25, text=white, font=\fontsize{9pt}{12pt}\selectfont, align=left] at (16.8,6.4) {数据层\\PostgreSQL / Redis / 对象存储\\评分缓存 + 指标索引};

  % 右列:数据与可视化
  \path[rounded corners=10pt, fill=white, fill opacity=0.07, draw=white!30]
    (20.4,5.2) rectangle (28.8,16.6);
  \node[anchor=north west, text=white, font=\bfseries\fontsize{14pt}{16pt}\selectfont] at (20.8,16.4)
    {数据可视化};

  % 条形图
  \node[anchor=west, text=white!90, font=\bfseries\fontsize{10pt}{12pt}\selectfont] at (20.8,15.6)
    {OpenAI 基准(平均分)};
  \def\barX{20.9}
  \def\barY{12.3}
  \def\barMax{6.2}
  \foreach [count=\i] \name/\val in {gpt-4o-mini/87.2,gpt-4o/86.7,o1/86.5,gpt-4-turbo/86.0,gpt-4/86.0} {
    \pgfmathsetmacro{\h}{\val/100*\barMax}
    \path[fill=cyan!45!white, draw=white!15, rounded corners=2pt]
      (\barX,\barY-1.1*\i) rectangle (\barX+\h,\barY-1.1*\i+0.8);
    \node[anchor=west, text=white, font=\fontsize{8pt}{10pt}\selectfont]
      at (\barX+0.15,\barY-1.1*\i+0.4) {\name\ (\val)};
  }
  \draw[white!25] (\barX,\barY-5.7) -- ++(\barMax,0);
  \node[anchor=west, text=white!60, font=\fontsize{8pt}{10pt}\selectfont] at (\barX,\barY-6.3) {金属哑光 + 生物荧光点光};

  % 雷达图
  \node[anchor=west, text=white!90, font=\bfseries\fontsize{10pt}{12pt}\selectfont] at (20.8,9.4)
    {VULCA 文化理解(Qwen2.5-VL vs Llama-Scout)};
  \def\cx{24.6}
  \def\cy{8.2}
  \def\r{3}
  \foreach \angle/\label in {90/语义贴合,210/人物对齐,330/文化理解} {
    \draw[white!25] (\cx,\cy) -- ++(\angle:\r);
    \node[text=white!75, font=\fontsize{8pt}{10pt}\selectfont] at ($( \cx,\cy ) + (\angle:\r+0.35)$) {\label};
  }
  \path ( \cx,\cy ) ++(90:\r*0.82) coordinate (Q1);
  \path ( \cx,\cy ) ++(210:\r*0.76) coordinate (Q2);
  \path ( \cx,\cy ) ++(330:\r*0.79) coordinate (Q3);
  \path[fill=cyan!50, draw=cyan!80, opacity=0.52] (Q1) -- (Q2) -- (Q3) -- cycle;
  \path ( \cx,\cy ) ++(90:\r*0.78) coordinate (L1);
  \path ( \cx,\cy ) ++(210:\r*0.71) coordinate (L2);
  \path ( \cx,\cy ) ++(330:\r*0.74) coordinate (L3);
  \path[fill=yellow!35, draw=yellow!80, opacity=0.45] (L1) -- (L2) -- (L3) -- cycle;
  \node[anchor=west, text=white!80, font=\fontsize{8pt}{10pt}\selectfont] at (26.8,9.0)
    {Qwen2.5-VL:0.82 / 0.76 / 0.79};
  \node[anchor=west, text=white!70, font=\fontsize{8pt}{10pt}\selectfont] at (26.8,8.3)
    {Llama-Scout:0.78 / 0.71 / 0.74};
  \node[anchor=west, text=white!60, font=\fontsize{8pt}{10pt}\selectfont] at (26.8,7.6)
    {构图:Isometric + Rule of thirds;光影:Soft studio lighting};

  % 底部 CTA
  \path[rounded corners=10pt, fill=cyan!55, draw=white!70] (1.2,1.0) rectangle (28.8,1.8);
  \node[anchor=west, text=black, font=\bfseries\fontsize{10pt}{12pt}\selectfont] at (1.6,1.3)
    {体验:前端 http://localhost:5173 | 后端 http://localhost:8001 | 演示账号 demo/demo123};
  \node[anchor=east, text=black, font=\fontsize{9pt}{11pt}\selectfont] at (28.4,1.3)
    {更多:2025.findings-emnlp.103.pdf 摘要、openai\_benchmark\_v2\_report.md};
\end{tikzpicture}
\end{document}
